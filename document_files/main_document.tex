%! Author = domore
%! Date = 8/11/21

% Preamble
\documentclass[11pt]{article}

% Packages
\usepackage{amsmath}
\usepackage{graphicx}
\usepackage{hyperref}
\hypersetup{
    colorlinks=true,
    linkcolor=blue,
    filecolor=magenta,
    urlcolor=cyan,
    citecolor=blue
}
\usepackage{cleveref}
\usepackage[margin=1.1in]{geometry}
\usepackage{python}
\usepackage{pythontex}
\usepackage{pdfpages}
% Documentation
% https://raw.githubusercontent.com/gpoore/pythontex/master/pythontex/pythontex.pdf
% tricks
% https://raw.githubusercontent.com/gpoore/pythontex/master/pythontex_gallery/pythontex_gallery.tex
% compilation
%pdflatex -aux-directory=/home/domore/PycharmProjects/ds_project-1/document_files/aux -output-directory=/home/domore/PycharmProjects/ds_project-1 main_document.tex
%python document_files/pythontex/pythontex3.py document_files/main_document.tex
%pdflatex -aux-directory=/home/domore/PycharmProjects/ds_project-1/document_files/aux -output-directory=/home/domore/PycharmProjects/ds_project-1 main_document.tex
\usepackage[framemethod=TikZ]{mdframed}
\usepackage{booktabs}
%\usepackage{multirow}
\usepackage{rotating}

% for future projects
% \usepackage[usefamily=juliacon]{pythontex}

% Document
\begin{document}

\includepdf[pages={2}]{src/titlepage.pdf}

\tableofcontents\newpage

\part{Wine quality} \label{part:wine}

\section{Introduction}\label{sec:introduction}
In the paragraphs to come we will discuss different approaches and models to be used in the dataset used in~\cite{wine}.

The following dataset consist of different a sample of wines with different characteristics and their relevant
quality.
The box below has an overview description of the dataset we will analyse.
A further description of each variable can be found in \Cref{tab:description_wine}.

% these lines will not be presented in the pdf
% pyconcode only excecutes (does not show code)
%! suppress = EscapeUnderscore
%! suppress = EscapeHashOutsideCommand
\begin{pyconcode}
import pandas as pd
import os
import numpy as np

# Referencing folders and data names
path = os.getcwd()
# We have to re-structure the path since we are in LaTeX
path = os.path.abspath(os.path.join(path, os.pardir))
file_name = 'winequality-red.csv'
path_file = f'{path}/code_python/project1_wine/data/{file_name}'
# We load the data and present an overview
wine_data = pd.read_csv(path_file)
\end{pyconcode}

%! suppress = EscapeUnderscore
%\begin{pyconsole}[][frame=single]
\begin{pyconsole}[][]
wine_data.info()
wine_data.describe().round(decimals=2).transpose()
\end{pyconsole}

\section{Exploratory analysis}\label{sec:exploratory-analysis}

To avoid unnecessary analysis, first we will perform a few checks to get a deeper understanding of the
data we are using.
To that end, we first check that the correlation structure amongst the variables (see \cref{fig:wine_heatmap}),
including the quality of the wine.
This should give us a general idea of how and if the variables are related to one another.


\begin{figure}[h!]
    \includegraphics[width=0.9\textwidth]{figs/wine_heatmap}
    \caption{Correlation structure of wine characteristics.}
    \label{fig:wine_heatmap}
\end{figure}


There are a few things we could check in the data to assess whether what we are looking relates somehow
to our prior knowledge about the topic.
This prior knowledge might help us identify certain links, that might not be obvious if the data were not labelled.

At first glance, one can note that \emph{fixed acidity} and \emph{citric acidity} are strongly correlated (negatively) to the \emph{pH},
although their correlation is not \emph{-1}.
This should relate to our prior knowledge, given that \emph{pH} is directly related to the acidity of a
solution.
Additionally, we can see that \emph{alcohol} correlates negatively with the \emph{density} of the wine.
This makes sense, given that the wine is a solution of different solutes, those in all likelihood are ``heavier''
than the alcohol solute, as well as the water solvent which is definitely ``heavier'' than alcohol.
Hence, the more the alcohol content increases in the wine, the less dense it becomes.
Another interesting characteristic of the wine that is highly correlated to its quality is the
alcohol content.

\vspace{2pt}
Another visual analysis we could perform are KDEs (Kernel Density Estimates).
These, we could imagine as a cross-sectional cut in a bi-variate probability distribution.
\Cref{fig:wine_kdes} shows us that even though the wines in the sample range from quality 3 to 8
(see histogram subplot in row 3, column 4), it would seem that there are mostly 4 important groups.
The earlier stated fact, ad priori, provide us with a key insight we should have in mind when trying
to fit any kind of model.
I.e., the tails of the quality (grades 3 and 8) will be underrepresented, then most models we could think
of fitting will have trouble predicting a grade close to 3 and 8 and beyond (to each direction).

\begin{figure}[h!]
    \includegraphics[width=0.9\textwidth]{figs/wine_kde}
    \caption{KDEs for wine features.}
    \label{fig:wine_kdes}
\end{figure}


\section{Analyses}\label{sec:analyses}

Among the many features that we could analyse in these data\footnote{Some other analyses, like clustering the wine sample do
carry the same importance. I.e., we could try clustering the wine sample, to find that certain
wines belong or were produced by the same vineyard due to the similarities in the grapes which
will most likely relate to their characteristics when turned into wine.
Even though this might be interesting, it is not as useful to a researcher/data scientist trying to add value
to the winemaking processes.}, we will stick to predicting the quality
of a wine given its characteristics.

To predict the quality of the wines, we will use all possible available features\footnote{Could be that using all
features decreases the predictive power of certain models. But these we could only find out through thorough investigation
which is out of the scope in this project.} and split our database: 80\% of the data will be used for training and
the remaining 20\% will be used for testing and computing assessment metrics.
To achieve this, we will mostly use \emph{sci-kit learn} library and sub-libraries.

%! suppress = EscapeUnderscore
%! suppress = EscapeHashOutsideCommand
\begin{pyconsole}[][]
# Pre-processing
# Shaping data
X = wine_data.drop('quality', axis=1)
y = wine_data['quality']
# Defining a seed for shuffling the data
seed = 123
# Splitting data into train/test.
from sklearn.model_selection import train_test_split
X_train, X_test, Y_train, y_test = train_test_split(X, y, test_size=0.20,
                                                    random_state=seed)
\end{pyconsole}

After splitting our initial sample into train and test, we proceed to standardize it.
Although this is not absolutely necessary, there are some models we will use in the following that
have better performance and provide better estimates\footnote{Less prone to be biased.}.
Then, some models will use the standardized data, and some others (like in \emph{tree regression})
will not.

%! suppress = EscapeUnderscore
%! suppress = EscapeHashOutsideCommand
\begin{pyconsole}[][]
# Converting 1D arrays to dataframe
y_train = pd.DataFrame(Y_train, columns=['quality'])
# Scaling training samples
from sklearn.preprocessing import StandardScaler
train_X_scaler = StandardScaler()
train_Y_scaler = StandardScaler()
# we first fit the training data to different scaling objects
# to keep track of them
train_X_scaler.fit(X_train)
train_Y_scaler.fit(y_train)
x_train = train_X_scaler.transform(X_train)
y_train = train_Y_scaler.transform(y_train)
\end{pyconsole}

It is important to note that the scaling (standardization) is only performed in the \textbf{training sample}.
Then, when we have our predictions from the models in the \emph{scaled space}, we will revert them back
to the \emph{observation space} using the inverse scaling we determined on the train sample.
This is of high importance when trying to build a predictive model.
If we used the scaling from the test sample, we would be \emph{snooping into the data}, which would probably
increase the models' accuracy (artificially), but with the power of \emph{hindsight}.

In the following snippets, code of the fitting will be provided.
After all models are presented, the results will be shown, and some discussion stemming from them will be
held.

\subsection{Linear Regression}\label{subsec:linear-regression}

%! suppress = EscapeUnderscore
%! suppress = EscapeHashOutsideCommand
\begin{pyconsole}[][]
# Linear Regression
# Normal Linear Regression (L2 Norm)
from sklearn.linear_model import LinearRegression as lr

# We first initialise the model and then fit with the training obs.
normal_lr = lr(fit_intercept=True, normalize=False)
normal_lr.fit(X=x_train, y=y_train)
# Predicting values, we first need to transform the X_test matrix
# using our earlier defined scale
nlr_y = normal_lr.predict(train_X_scaler.transform(X_test))
# Reverting our predicted values to their level using the scale determined
# from the training sample
nlr_y = train_Y_scaler.inverse_transform(nlr_y)
\end{pyconsole}

\subsection{Lasso Regression}\label{subsec:lasso-regression}
%! suppress = EscapeUnderscore
%! suppress = EscapeHashOutsideCommand
\begin{pyconsole}[][]
# Lasso Linear Regression (L1 Norm)
# we do not standardize here otherwise the lasso regression might turn
# all coefficients to 0 only to use the intercept.
from sklearn.linear_model import Lasso as lasso

# We first initialise the model and then fit with the training observations
# also do not use intercept, if the intercept is allowed, again makes
# most of the coefficients to be 0.
lasso_lr = lasso(fit_intercept=False, normalize=False)
lasso_lr.fit(X=X_train, y=Y_train)
lassolr_y = lasso_lr.predict(X_test)
\end{pyconsole}
\newpage
\subsection{Neural Networks}\label{subsec:neural-networks}

%! suppress = EscapeUnderscore
%! suppress = EscapeHashOutsideCommand
\begin{pyverbatim}[][]
# Neural Network - using sklearn
from sklearn.neural_network import MLPRegressor

NN_scikit = MLPRegressor(random_state=seed,
                         max_iter=500,
                         hidden_layer_sizes=(22,),
                         activation='logistic').fit(x_train,
                                                    np.ravel(y_train))
NN_scikit.out_activation_ = 'identity'
# we use (22,) in the hidden layers to try to capture the features and
# their interactions, we will do it because it is too little data.
# Using more neurons or hidden layers in this case might induce
# over-fitting, given the small sample size.
nn_scikit_y = NN_scikit.predict(train_X_scaler.transform(X_test))
# Reverting our predicted values to their level using the scale determined
# from the training sample.
nn_scikit_y = train_Y_scaler.inverse_transform(nn_scikit_y)
\end{pyverbatim}

%! suppress = EscapeUnderscore
%! suppress = EscapeHashOutsideCommand
\begin{pyverbatim}[][]
# NN With Keras
from keras.models import Sequential
from keras.layers import Dense
import tensorflow as tf
tf.random.set_seed(seed)


def squared_loss(y_true, y_pred):
    squared_difference = tf.square(y_true - y_pred)
    return tf.reduce_sum(squared_difference, axis=-1)  # Note the `axis=-1`


nn_keras_tf = Sequential()
nn_keras_tf.add(Dense(22,
                      input_dim=11,  # expects 11 inputs
                      activation='sigmoid'))
nn_keras_tf.add(Dense(1, activation='linear'))  # output layer
# since we are using keras to call TF we need to compile the model we
# designed in keras for it to be into the TF framework.
# print(dir(tf.keras.optimizers))
# print(dir(tf.keras.losses))
# print(dir(tf.keras.metrics))
nn_keras_tf.compile(loss=squared_loss, optimizer='adam')
nn_keras_tf.fit(x_train, y_train, epochs=500)
nn_keras_tf_y = train_Y_scaler.inverse_transform(
                nn_keras_tf.predict(train_X_scaler.transform(X_test)))
\end{pyverbatim}

\subsection{Regression Tree}\label{subsec:regression-tree}

%! suppress = EscapeUnderscore
%! suppress = EscapeHashOutsideCommand
\begin{pyverbatim}[][]
# Regression Tree
from sklearn.tree import DecisionTreeRegressor

tree_model = DecisionTreeRegressor(max_depth=4)
# Data without scaling.
tree_model.fit(X_train, Y_train)
tree_y = tree_model.predict(X_test)
\end{pyverbatim}

\subsection{Support Vector Regression (SVR)}\label{subsec:support-vector-regression-(svr)}

%! suppress = EscapeUnderscore
%! suppress = EscapeHashOutsideCommand
\begin{pyverbatim}[][]
# SVM (Regression)
from sklearn.svm import SVR

svr = SVR()
svr.fit(x_train, np.ravel(y_train))
svr_y = train_Y_scaler.inverse_transform(svr.predict(
                                        train_X_scaler.transform(X_test)))
\end{pyverbatim}


\section{Results}\label{sec:results}

Most of the models used in \Cref{sec:analyses} were fitted using \emph{sk-learn} libraries.
This was done just for ease of quick prototyping and implementation, but for production models, probably
other libraries would be preferred, if not outright developing some of them from scratch.
One model, one neural network, was additionally fitted with the \emph{keras} library to emphasize that the results
should be quite similar, however not identical\footnote{There are several reasons why almost two identically defined
models can have different outputs, in particular when it comes to neural networks. Some of these factors are the
solvers used, the number of epochs, learning rate and steps, tolerances and staring points.}.

%\caption{Description of wine characteristics.}
\begin{table}[h!]
\centering
\caption{Metric comparison different competing models.}
\label{tab:model_results}
\begin{tabular}{lrrr}
\toprule
{} &  RMSE &  MAPE &  Score \\
\midrule
linear regression       &  0.66 &  0.10 &   0.34 \\
lasso linear regression &  0.73 &  0.10 &   0.20 \\
NN Scikit               &  0.65 &  0.10 &   0.36 \\
NN Keras TF             &  0.64 &  0.09 &   0.38 \\
Regression Tree         &  0.69 &  0.10 &   0.28 \\
SVR                     &  0.61 &  0.08 &   0.44 \\
\bottomrule
\end{tabular}
\end{table}


%\label{tab:model_results}

\Cref{tab:model_results} shows the results of different loss functions and score for the different competing models.
Let us quickly define each of the metrics.

\begin{equation}
  \mathrm{RMSE} = \sum_{i=1}^{d}
  \left(\frac{y_{i}-\hat{y}_{i}}{y_{i}}
  \right)^2.
\label{eq:rmse}
\end{equation}

\begin{equation}
  \mathrm{MAPE} = \sum_{i=1}^{d}
  \left|\frac{y_{i}-\hat{y}_{i}}{y_{i}}
  \right| \cdot 100\%.
\label{eq:mape}
\end{equation}

\begin{equation}
  \mathrm{Score} = 1 - \frac{\sum_{i=1}^{d}(y_{i}-\hat{y}_{i})^2}{\sum_{i=1}^{d}(y_{i}-\bar{y})^2}.
\label{eq:score}
\end{equation}

Where $y_{i}$ is the true $i^{\text{th}}$ observation in the test sample, $\hat{y}_{i}$ is the forecasted
$i^{\text{th}}$ observation and $\bar{y}$ denotes the average value of the test sample.

After observing the values in \cref{tab:model_results}, one can see a clear winner in terms of the loss
metrics, i.e.\ the SVR model which has the lowest RMSE and MAPE out of the competing models as well as
the largest Score out of the models in scope.

Another fact we can observe is that the MAPE is $0.1$ for most of the models, although this is just due the
rounding of results to 2 decimal places.

Also, if we compared the two more classic regressions: Linear regression, and the \emph{lasso} regression,
we find that it would seem the latter to be of inferior performance than the former.
This was to be expected, given that the \emph{lasso} regression penalises the amount of regressors when
more systematic error could still be extracted by using more regressors.\ By the same token, the \emph{lasso}
regression looks to find a sparser model, which is useful when one tries to make sense of a model with fewer
variables, and hence easier to control for and to explain.

When comparing the neural network models, we can observe that both models perform quite similarly.
They do not output exactly the same results even when trying to match their parametrisation and hyper-parameters,
as one can observe in \cref{subsec:neural-networks}.
This is also to be expected from models like neural networks, since they are initialised (pseudo)randomly, hence
the (stochastic) gradient descent algorithm can output widely different results with the same inputs, just by
factors like starting points (in particular when the starting points are \emph{close} in \emph{input space} to a
local minimum) and initial shuffling of the sub-samples to initialise the algorithm.




\section{Discussion}\label{sec:discussion}

So far, we have covered quantitatively the results in terms of the loss metrics.
If we had stopped here, then we should conclude that the SVR model was superior to the rest
and therefore use this model in the future.

Let us have a look at the distribution of the predictions of each model and compare them to the
distributions

% todo describe the distributions and add the (probability) histograms

% todo make a SHAP analysis on one of the models

% todo explain how to bypass the issue of the distribution tails when forecasting on new data either ensamble models or local approximations.


{\color{red}\textbf{[1]}} All these characteristics are important in a wine.
We used all these statistical models to try to understand to which degree they are important
to determine a wine`s quality.
These analyses might be really important for a winemaker to know.
Given that by affecting the inherent characteristics of the wine will most likely have an
impact on the quality of that wine.



\begin{thebibliography}{1}
\bibitem{wine} P. Cortez, A. Cerdeira, F. Almeida, T. Matos and J. Reis. {\em Modeling wine preferences by data mining from physicochemical properties.
In Decision Support Systems, Elsevier, 47(4):547-553},  2009.
\end{thebibliography}


\begin{sidewaystable}
    \centering
    \caption{Description of wine characteristics.}
    \resizebox{\columnwidth}{!}{
    \input{../code_python/project1_wine/description_wine.tex}
    }
    \label{tab:description_wine}
\end{sidewaystable}





\end{document}